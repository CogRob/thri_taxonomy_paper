\documentclass[letterpaper, 10 pt, conference]{ieeeconf}  % Comment this line out
                                                          % if you need a4paper
%\documentclass[a4paper, 10pt, conference]{ieeeconf}      % Use this line for a4
                                                          % paper

\IEEEoverridecommandlockouts                              % This command is only
                                                          % needed if you want to
                                                          % use the \thanks command
\overrideIEEEmargins
\let\labelindent\relax

% See the \addtolength command later in the file to balance the column lengths
% on the last page of the document



% The following packages can be found on http:\\www.ctan.org
\usepackage{graphics} % for pdf, bitmapped graphics files
%\usepackage{epsfig} % for postscript graphics files
%\usepackage{mathptmx} % assumes new font selection scheme installed
%\usepackage{times} % assumes new font selection scheme installed
%\usepackage{amsmath} % assumes amsmath package installed
%\usepackage{amssymb}  % assumes amsmath package installed
\let\proof\relax
\let\endproof\relax
\usepackage{amsthm}
\theoremstyle{definition}
\newtheorem{definition}{Definition}
\graphicspath{{images/}}
\usepackage{xcolor,soul}
\usepackage[inline, shortlabels]{enumitem}
\usepackage{biblatex}
\renewcommand*{\bibfont}{\small}
% personal productivity
\usepackage[colorinlistoftodos]{todonotes}
\presetkeys{todonotes}{inline}{}
\usepackage{hyperref}
\newcommand{\toread}[1]{\todo[color=red!40!white]{#1}}
\newcommand{\towrite}[1]{\todo[color=blue!40!white]{#1}}
\renewcommand{\hl}[1]{{\color{red}#1}}
\AtEveryBibitem{%
\ifentrytype{book}{
    \clearfield{url}%
    \clearfield{doi}%
    \clearfield{issn}%
    \clearfield{urldate}%
    \clearfield{review}%
    \clearfield{series}%%
}{}
\ifentrytype{collection}{
    \clearfield{url}%
    \clearfield{doi}%
    \clearfield{issn}%
    \clearfield{urldate}%
    \clearfield{review}%
    \clearfield{series}%%
}{}
\ifentrytype{incollection}{
    \clearfield{url}%
    \clearfield{doi}%
    \clearfield{issn}%
    \clearfield{urldate}%
    \clearfield{review}%
    \clearfield{series}%%
}{}
\ifentrytype{article}{
    \clearfield{url}%
    \clearfield{doi}%
    \clearfield{issn}%
    \clearfield{urldate}%
    \clearfield{review}%
    \clearfield{series}%%
}{}
\ifentrytype{inproceedings}{
    \clearfield{url}%
    \clearfield{doi}%
    \clearfield{issn}%
    \clearfield{urldate}%
    \clearfield{review}%
    \clearfield{series}%%
}{}
\ifentrytype{techreport}{
    \clearfield{url}%
    \clearfield{doi}%
    \clearfield{issn}%
    \clearfield{urldate}%
    \clearfield{review}%
    \clearfield{series}%%
}{}
\ifentrytype{misc}{
    \clearfield{url}%
    \clearfield{doi}%
    \clearfield{issn}%
    \clearfield{urldate}%
    \clearfield{review}%
    \clearfield{series}%%
}{}
}

\bibliography{rexam}

\title{\LARGE \bf
A Taxonomy for Characterizing Modes of Interactions in Goal-driven, Human-robot Teams
}
\newcommand{\citet}[1]{\citeauthor{#1}~\cite{#1}}
%\author{ \parbox{3 in}{\centering Huibert Kwakernaak*
%         \thanks{*Use the $\backslash$thanks command to put information here}\\
%         Faculty of Electrical Engineering, Mathematics and Computer Science\\
%         University of Twente\\
%         7500 AE Enschede, The Netherlands\\
%         {\tt\small h.kwakernaak@autsubmit.com}}
%         \hspace*{ 0.5 in}
%         \parbox{3 in}{ \centering Pradeep Misra**
%         \thanks{**The footnote marks may be inserted manually}\\
%        Department of Electrical Engineering \\
%         Wright State University\\
%         Dayton, OH 45435, USA\\
%         {\tt\small pmisra@cs.wright.edu}}
%}

\author{
Priyam Parashar, Lindsay M. Sanneman, Julie A. Shah, Henrik I. Christensen
}


\begin{document}



\maketitle
\thispagestyle{empty}
\pagestyle{empty}


%%%%%%%%%%%%%%%%%%%%%%%%%%%%%%%%%%%%%%%%%%%%%%%%%%%%%%%%%%%%%%%%%%%%%%%%%%%%%%%%
\begin{abstract}

As robots and other autonomous agents are incorporated into increasingly complex domains, classifying interaction in heterogeneous teams including both humans and automation is necessary. Previous literature has addressed the task of classifying human-robot interaction from different aspects and situated in different contexts. However, the factors behind an interaction work in unison and the insights from one perspective inadvertently affect the other. This motivates a need for unification of these taxonomies and framework within an upper ontology which can systematically define these relationships. In this paper we review existing taxonomies from human-robot interaction, the behavioral sciences, social taxonomies, and algorithmic taxonomies and propose an upper ontology over these works. We identify three main components characterizing interaction systems: task, environment, and team, and structure them over two levels: contextual factors and factors driven by local dynamics. Finally, we identify a classification of the effects given contextual factors and local dynamics and discuss open areas of research motivated by the observed gaps in the literature. 

\end{abstract}


%%%%%%%%%%%%%%%%%%%%%%%%%%%%%%%%%%%%%%%%%%%%%%%%%%%%%%%%%%%%%%%%%%%%%%%%%%%%%%%%

\section{INTRODUCTION}

\section{Application of the Taxonomy to Real-World Applications}
In the previous section, we introduce a taxonomy over factors impacting interaction in heterogeneous teams. These include contextual factors which cover characteristics of problem inputs, dynamics factors which include different characteristics of decisions made in terms of problem approach, and effects which are the resultant requirements on interaction given the context and dynamics. In this section we introduce a design process, drawing from the Cognitive Work Analysis (CWA) and co-active design literature, by which contextual factors can be linked to the decision of dynamics factors which in turn influence the effects. The overview of this process is depicted in figure \ref{fig:tax_design}. 

    \begin{figure}[tb]
    \centering
    \includegraphics[width=\columnwidth]{TaxonomyDesign.png}
   \caption{Process for Determining Dynamics Factors and Effects}
    \label{fig:tax_design}
    \end{figure}

Following from work in co-active design, steps in this process include creation of an abstraction hierarchy, determination of observability, comprehensibility, predictability, and directability (OCPD) requirements, and interdependence analysis. Contextual factors act as inputs to all steps in the process, and decision of dynamics factors and effects are the outputs. We first detail each of these steps and then present two domain examples including a rover planning scenario and an autonomous vehicle scenario to demonstrate the application of our overall taxonomy and this process. In the discussion of each of the steps in our proposed process, we will draw on a capture-the-flag example scenario.

In the capture-the-flag domain, we consider a scenario in which two teams each protect a flag in their own territory while attempting to capture the other team's flag in that team's territory. Team members who enter into the enemy team's territory and are tagged by an enemy team member are eliminated from the game. The first team to capture the enemy team's flag and successfully return it to their own territory wins the game. We ground the capture-the-flag scenario in the RoboFlag testbed domain introduced by D`Andrea and Babish [cite this]. In their setup, two teams of 6-10 robots and 2 people apiece in a hybrid simulated-physical environment are each trying to cross into the opponent’s territory, capture their flag, and return to their own territory while evading the enemy team. Some of the primary challenges this domain poses are that teammates have limited sensing capabilities, there is distributed processing of information, and there are limited bandwidth capabilities.


\subsection{Abstraction Hierarchy}
The CWA literature introduces the idea of an Abstraction Hierarchy (AH) for decomposition of domain goals and tasks required to support those goals for a given mission or high-level task [CITE CWA literature + applied decision science]. Goals and tasks are represented at different levels of abstraction including mission goals, priorities and values, generalized functions, and temporal functions. An example of the top three levels of the Abstraction Hierarchy (excluding the lowest-level temporal functions) for the capture-the-flag scenario is shown in Figure \ref{fig:ctfah}. Examples of temporal functions at the lowest level of the Abstraction Hierarchy for the capture-the-flag scenario are detailed in examples in sections \ref{sec:OCPD} and \ref{sec:IA}.

   
    \begin{figure*}[tb]
    \centering
    \includegraphics[width=\textwidth]{ctfAH.png}
   \caption{Abstraction Hierarchy for Capture the Flag Scenario}
    \label{fig:ctfah}
    \end{figure*}

We propose development of an Abstraction Hierarchy as the first step of the analysis process for determining dynamics factors and resultant effects from the inputted contextual factors. As heterogeneous teams of humans and autonomous agents aim to achieve increasingly complex tasks in complex domains, different team members will need information at different levels of abstraction. The Abstraction Hierarchy provides a framework from which to determine different possible levels of abstraction of information for communication between teammates or sub-teams. 

%Contextual factors inform the definition of the abstraction hierarchy at all levels. In particular, contextual task contextual factors including requirement type and task focus and contextual environment type factors including expected dynamics and hazard level should be considered when developing the hierarchy. Task requirement type will inform the definition of generalized functions and temporal functions, task focus will inform the definition of mission goals, priorities and vales, and functions, expected dynamics will inform the definition of temporal functions, and hazard level will inform the definition of priorities and values. 

Contextual factors inform the definition of the abstraction hierarchy at the lower three levels, with the mission goals being taken as input and defining the top level. In particular, contextual task factors including task focus and task critically should be considered when developing the hierarchy. Task focus will inform the definition of generalized functions and temporal functions. For example, in the capture-the-flag scenario, the generalized function ``Create protective barrier around flag'' is an ``Area Coverage'' problem, ``Attack opponents in our territory'' is a ``Target Search'' problem, and so on. Task criticality will inform the definition of priorities and values. For example, For example, for ``high'' or ``severe'' criticalities in which human-safety is a concern, ``Safety'' should be included as a box in the priorities and values level. In the capture-the-flag scenario, we assume that tasks are only mission-critical, so mission-related priorities and values including ``Protect our flag'' and ``Capture the other team's flag'' are included in our priorities and values. Each, generalized function is associated with a particular priority or value or multiple of them.

%TODO:Maybe add something about the lowest level begin influenced by dynamics factors? Not sure about this yet.

%task focus will inform the definition of mission goals, priorities and vales, and functions, expected dynamics will inform the definition of temporal functions, and hazard level will inform the definition of priorities and values. 


In conjunction with analysis of observability, comprehensibility, predictability, and directability requirements and the interdpendence analysis, task allocation can be done at different levels of abstraction depending on team member capabilities through consideration of the Abstraction Hierarchy. Assigning tasks at a higher level of abstraction allows teams or individuals greater flexibility in how they execute tasks, while assigning tasks at a lower level of abstraction provides teams or individuals more detailed instructions as how to how execute a task when necessary. In other words, there is a trade-off between pre-planning and flexibility in the task allocation aspect of mission planning.

\subsection{Interdependence Analysis}
\label{sec:IA}
[cite CAD] proposes an Interdependence Analysis (IA) process to determine team decomposition and task allocation. Specifically, they introduce the concept of an IA table in which capacities required for each task and considered in conjunction with team member capabilities to determine different possible team structures for approaching a joint task. 
%While in their work, OPD requirements are derived from the connections between subtasks, we see OCPD requirements as part of the capacities required for each task as determined by the abstraction hierarchy and include them as requirements in the IA table. 
The output of the IA process is the enumeration of various possible team decompositions with corresponding task allocations that are feasible assignments for task execution as well as the OCPD requirements for each team configuration. Drawing from the the various levels of abstraction in the abstraction hierarchy, when performing the interdependence analysis, tasks can be considered at different levels of abstraction as necessary. 

An example IA table, as is used in the process proposed in [cite CAD] is shown in figure \ref{fig:iatable_ctf}. In the table, we show an example analysis for the ``Create protective barrier around the flag'' generalized function. The hierarchical sub-tasks shown represent the tasks at the ``temporal functions'' (lowest) level of the abstraction hierarchy, which we do not include in figure \ref{fig:ctfah}. The required capacities break down each temporal function into even more discretized functional requirements. In our example, we assume we have a team of one robot and one human assigned to the ``create protective barrier'' function, and we can analyze how we can allocate the low-level subtasks to them in order to accomplish the higher level goal of flag protection. We follow the coloring scheme adopted by [cite CAD], shown in figure [REF].

Development of the IA table leverages various contextual factors and helps system designers determine a number of dynamics factors for mission execution. Task nature helps define what the required capacities are for the hierarchical sub-tasks listed in the table. The team composition and team capabilities define which team members are available to complete the various tasks and sub-tasks for the mission and what capabilities they have to perform tasks on their own or in conjunction with other team members. The team-related contextual factors determine the color-coding in the IA table for the various team decomposition alternatives. This analysis can and should be performed at multiple levels of abstraction in order to best understand how to allocate team members to tasks at different levels of the abstraction hierarchy (in other words, to best understand how to allocate sub-teams within the larger team). 
%TODO: Could add specific example from our CTF IA table above.

The process of developing the IA table and considering various alternative team and sub-team decompositions results in the determination of the following dynamics factors: plan-time dependencies, run-time dependencies, team roles, and the expertise hierarchy. The chosen task allocation will determine schedule dependencies between various teammates, defined by the plan-time dependency categories. Similarly, the task decomposition that is chosen will dictate whether sub-goals are joint or disjoint and whether tasks should be done in parallel, sequentially, or in a dialog-like manner (run-time dependencies). Finally, the IA analysis determines the particular roles each team member plays and defines the expertise hierarchy (whether decisions are made in a centralized way and then communicated to other team members or whether decisions are made in a distributed way; whether the decomposition should be fixed or should be allowed to be fluid). 
%TODO: Again, maybe give specific examples above. 

    \begin{figure*}[tb]
    \centering
    \includegraphics[width=\textwidth]{IATable-CtF.png}
   \caption{Example Interdependence Analysis Table for Capture the Flag Scenario}
    \label{fig:iatable_ctf}
    \end{figure*}

\subsection{Observability, Comprehensibility, Predictability, and Directability Requirements}
\label{sec:OCPD}
[Cite co-active design] further propose the consideration of observability, predictability, and directability (OPD) requirements in the determination of how a task should be structured for a joint activity. These are requirements that go beyond task-work requirements and are necessary for the support of accomplishment of a joint activity through mutual understanding and influence between teammates. 

[Cite CAD] define observability as making pertinent aspects of one’s status, as well as one’s knowledge of the team, task, and environment observable to others. They define predictability as the need for one’s actions to be predictable enough that others can reasonably rely on them when considering their own actions. Finally, they define directability as one’s ability to direct the behavior of others and complementarily be directed by others. 

Drawing from the definition of the three levels of situational awareness in [cite Endsley], perception, comprehension, and projection, we see analogs between [CAD's] observability and predictability and [Endsley's] perception and projection, respectively. However, while comprehension is a critical component of awareness in a situation, there is currently no analog to comprehension in the OPD framework. We therefore expand on the OPD framework to additionally consider ``comprehensibility'' requirements as an additional category, giving us an OCPD framework. We define comprehensibility here similar to how [cite Endsley] does: one's understanding of the meaning of observed elements in the environment and how they relate to situational goals. 

We propose consideration of OCPD requirements as the third step in the analysis process for determining dynamics factors and effects. While the Abstraction Hierarchy breaks down tasks and task-work that can be assigned at different levels of abstraction and IA analysis explores different sub-team compositions and task allocations, the corresponding OCPD requirements for each task in the hierarchy determine what specific information needs to be known or communicated at each level of abstraction to support the task-work. OCPD requirements can be defined for each required capacity in the IA table as well as between required capacities in order to support this analysis. %maybe think about if this is the right way to do this (OCPD maybe only need to be defined for a subset)

An example set of OCPD requirements is shown in figure \ref{fig:ocpd_ctf}. Here we consider the ''Sense objects and identify as enemy team members'' required capacity. We assume that the human or the robot can sense moving objects in the environment, but only the human can identify the moving objects as enemy team members or not. This dictates the OCPD requirements described in the figure. Note that if team member role alternative 1 is chosen and the human performs enemy team member identification entirely independently (with sensing help from the robot), an observability requirement is created between the ``sense and identify'' capacity and the ``track enemy team member locations'' capacity, since the robot will need to be able to observe which objects the human has identified as enemy team members. 
%TODO: Maybe add in an arrow to show the cross-capacity OCPD requirements. 
%[TODO: ADD IN CTF EXAMPLE HERE -- choose an individual generalized function to analyze]

As with the definition of the abstraction hierarchy and IA analysis, contextual factors inform the development of the OCPD requirements for tasks at each level of abstraction, and the OCPD requirements help define dynamics factors. In defining OCPD requirements, the shared-knowledge structures contextual team factor can be used to assess what each team member reasonable knows a priori and what needs to be communicated between team members. The OPCD requirements then determine the team communication dynamics factor, since what needs to be observed, comprehended, predicted, and directed by each teammate dictates which communication strategies are possible for that information. 
%TODO: Add in additional context and dyanmics factors as we determine them. Maybe also add in specific reference to the CTF example.

%Note: need to add in something about observability requirements corresponding to both what needs to be observed and what needs to be observabile, etc 

    \begin{figure*}[tb]
    \centering
    \includegraphics[width=\textwidth]{OCPDReqs-CtF.png}
   \caption{OCPD Requirements Example for Capture the Flag Scenario}
    \label{fig:ocpd_ctf}
    \end{figure*}


%TODO: Maybe we no longer need this paragraph. Modify it to just discuss effects. 
\subsection{Tying Context to Dynamics and Effects}
The process of creating an abstraction hierarchy, performing an interdependence analysis, and determining OCPD requirements results in different possible team and task decompositions for the task that the team is aiming to accomplish. Once the team and task decomposition is determined, dynamics factors and effects can be considered. Task and goal dependencies, team roles, the expertise hierarchy, and communication are the outputs of the analysis process. The level of autonomy and level of information abstraction effects are also directly determined by the outputs of this process.


\newpage
\pagebreak
\clearpage

%%TO DO: Modify the text in the above section
\subsection{Domain Examples}
In the following section we detail three examples in which we apply our analysis method to a capture the flag domain, a rover planning domain, and an autonomous vehicle domain. 
\begin{enumerate}
%\item Capture the Flag Example Domain
%\\In the capture-the-flag domain, we consider a scenario in which two teams each protect a flag in their own territory while attempting to capture the other team's flag in that team's territory. Team members who enter into the enemy team's territory and are tagged by an enemy team member are eliminated from the game. The first team to capture the enemy team's flag and successfully return it to their own territory win the game. We ground the capture-the-flag scenario in the RoboFlag testbed domain introduced by D`Andrea and Babish {cite}. In their setup, two teams of 6-10 robots and 2 people apiece in a hybrid simulated-physical environment are each trying to cross into the opponent’s territory, capture their flag, and return to their own territory while evading the enemy team. This domain demonstrates human-robot interaction at high speeds in dynamic, unstructured environments. Some of the challenges this domain poses are that teammates have limited sensing capabilities, there is distributed processing of information, and there are limited bandwidth capabilities.

%The operational context that we use to ground our approach is the RoboFlag testbed domain introduced by D`Andrea and Babish \cite{d2003roboflag}. This domain demonstrates human-robot interaction at high speeds in dynamic, unstructured environments and thus provides a rich context for exploration of these topics. The RoboFlag testbed emulates a capture-the-flag scenario in which two teams of 6-10 robots and 2 people apiece in a hybrid simulated-physical environment are each trying to cross into the opponent’s territory, capture their flag, and return to their own territory while evading the enemy team. In this domain, the human and robot teammates need to interact at speed in an environment with limited sensing capabilities, distributed processing of information, and limited bandwidth capabilities In our analysis, we divide this scenario into ``scout" and ``defend" stages with subteams dedicated to each stage.
%Add text and citation from the workshop paper here.
\begin{enumerate}
    \item Contextual Factors: Problem Inputs
    \begin{itemize}
        \item Task Factors:\\
        \textit{Nature - Hybrid}\\
        The nature of the capture-the-flag task is a hybrid of cognitive and physical. The team must both plan their next actions based on the current situation, and they must execute their plan in the real world. \\
        \textit{Focus - Hybrid: Area Coverage, Target Search}\\
        In the capture-the-flag domain, task focus included both area coverage and target search. Team members must cover their own territory and protect it from enemy team members and must cover the enemy territory as they search for the flag. The overall task also has a target search focus, since the team members need to search for the target enemy flag. 
        \textit{Criticality - Medium Criticality}\\
        In the capture-the-flag scenario, we assume that human safety is not threatened at any point during execution. We designate this scenario as having medium criticaility, since the tasks that team members execute are mission critical in that failure to perform any of them could jeopardize the team's ability to capture the other team's flag or protect their own flag.\\
        \textit{Objective: Speed, accuracy, something else?}
        
        \item Team Factors:\\
        \textit{Composition - Multiple heterogeneous humans to multiple homogeneous robots}\\
        In the capture-the-flag scenario, grounded in the RoboFlag setup, there are 2 humans and 6 robots. All robots have identical capbilities, and we assume that the humans have varying capabilities. \\
        \textit{Capabilities - Motion: (navigation, grasping/manipulation, speed, etc.) Sensor: Computation: Communication: (about environment, about own behavior, about other teammates)}\\
        \textit{Modeling -}\\
        
        \item Environment Factors: \\
        \textit{Spatial Distribution - Hybrid}\\
        In the capture-the-flag scenario, the spatial distribution is a hybrid of proximate and remote. For some of the subtasks in capture-the-flag, such as protecting the flag, team members are proximate. In other subtasks, such as scouting and capturing the flag in enemy territory, team members are distributed over a larger area and must communicate with some other team members remotely. \\
        \textit{Level of Cooperation - Adversarial}\\
        The capture-the-flag environment is adversarial, since team members need to contend with the enemy team.\\
        \textit{Spatial and Temporal Complexity - Unstructured and Dynamic}\\
        The environment in the capture-the-flag scenario is unstructured in that team members may encounter unmapped obstacles in the enemy territory. Further the locations and movements of enemy team members are unstructured and unknown ahead of time. The environment is also dynamic in that the locations of enemy team members and obstacles can move over time during execution. \\
       \textit{Mobility and Perception Constraints - None}\\
       In the capture-the-flag domain, we assume that all team members' full mobility and perception abilities are available and usable within the environmental context. \\
       \textit{Other Factors that might be important: stress level (maybe in Robin Murphy paper), something about cognitive load for the human? (for robot, this might relate to just whether it can handle required computation capabilities), physical load for the human (like if it's the middle of the night, might be more tired)?, }
    \end{itemize}


    \item Abstraction Hierarchy
    \\The abstraction hierarchy for the capture the flag domain is shown in figure \ref{fig:ctfah}. We have detailed the first three levels of the hierarchy including mission goals, priorities and values, and generalized functions. We also detail a subset of the total functions at the temporal functions level. %%Note: need to add this to the picture%%
    
    The high-level mission goal in this scenario is to capture the opponent team's flag while preventing the opponent team from capturing your team's flag first. At the second level, the priorities and values are to ``protect our flag'' and ``capture the other team's flag''.
    

    \item Observability, Comprehensibility, Predictability, and Directability Requirements
    \\Here we detail the definition of OCPD requirements for a subset of the boxes at each level of the hierarchy to demonstrate how such requirements can be defined and used in analysis. We further use this subset in our interdependence analysis in section d.
    
    \item Interdependence Analysis
    %Include picture of interdependence analysis table here.
    %Need to discuss team member capabilities somewhere. Maybe in the input section.
    \item Tying Context to Dynamics and Effects
    \begin{itemize}
        \item Task Factors: \\
        \textit{Planning - Hybrid}\\
        In the capture-the-flag domain, task planning is a hybrid of online planning and offline planning. The team might make an overall strategy offline, before execution, but it also needs to be able to adapt to the environment with adversarial agents online, requiring online planning. \\
        \textit{Plan-time Dependencies - Complex Dependencies}\\
        In the 
        \textit{Run-time Dependencies - }\\
        \item Team Factors: \\
        \textit{Roles- }\\
        \textit{Expertise Hierarchy - }\\
        \textit{Communication Structure - }\\
    \end{itemize}
    Task:
		Planning - online (or hybrid?)
		Plan-time Dependencies - Complex dependencies
		Run-time Dependencies - dialog-like, sense-plan-act: joint

	Team:
		Roles - Operator or peer?
		Expertise Hierarchy -
			Reconfigurability - fixed
			Decision-making protocol - distributed

		Communication Structure - Not sure


\end{enumerate}


\item Rover Planning Example Domain
\\[Smith, 2012] introduces the Mars Exploration Rover activity planning domain in which a group of science and engineering sub-teams with potentially competing preferences work together to develop the rover’s tactical activity plan. Throughout the planning process, information is aggregated through complex coordination structures between sub-teams and a resultant time-consuming iterative planning process. [Smith, 2012] highlights the need to integrate automated planning into an iterative process that begins before goals, objectives, and preferences are fully defined and outlines the technical implications for planning, including the need to naturally specify and utilize constraints in the planning process, generate multiple qualitatively different plans for analysis, and provide explanation of planning decisions.

\begin{enumerate}
    \item Contextual Factors: Problem Inputs
    \item Abstraction Hierarchy
    \\The abstraction hierarchy for the rover planning domain is shown in figure \ref{fig:rpah}. As with the capture the flag example, we have detailed the first three levels of the hierarchy including mission goals, priorities and values, and generalized functions. We also detail a subset of the total functions at the temporal functions level. %%Note: need to add this to the picture%%
    
       \begin{figure}[tb]
    \centering
    \includegraphics[width=\columnwidth]{rpAH.png}
   \caption{Abstraction Hierarchy for Rover Planning Scenario}
    \label{fig:rpah}
    \end{figure}
    
    \item Observability, Comprehensibility, Predictability, and Directability Requirements
    \item Interdependence Analysis
    \item Tying Context to Dynamics and Effects
\end{enumerate}

\item USAR Domain

\begin{enumerate}
    \item Abstraction Hierarchy
    \item Observability, Comprehensibility, Predictability, and Directability Requirements
    \item Interdependence Analysis
    \item Tying Context to Dynamics and Effects
\end{enumerate}

\end{enumerate}

\newpage

\section{Other Thoughts on Categorizations}
\subsection{Task Factors - Context}
\textbf{Objective:} In the process of defining an abstraction hierarchy for a given mission, mission priorities and values are defined. The priorities and values support the overall mission goal and are supported by generalized functions, which are essentially mission tasks. The high-level objectives or priorities and values that tasks are supporting connect them to other tasks that support the same objectives, which might dictate how information-sharing needs to take place between individual or sub-teams executing each task. Objectives might also have different priorities, which can then determine the priorities of the tasks that support them relative to other tasks. Task priorities will impact how resources are allocated when dynamics factors are assigned. 

Some example objectives include:
\begin{itemize}
    \item Safety
    \item Efficiency
    \item Cost Minimization
    \item Learning
    \item Completeness (Coverage)
\end{itemize}

%(Beer paper talks about objectives I think - task focus) 

\subsection{Environment Factors - Context}
\textbf{Stress Level:} While autonomous agents are not impacted in their ability to perform tasks based on the level of stress in a given context, human teammates are impacted by stress. One example of this is that a human maybe have more difficulty performing tasks late at night, with little sleep, or if it is an emergency scenario. We define three categories for stress levels accordingly:
\begin{itemize}
    \item Low Stress
    \item Normal Stress
    \item High Stress
\end{itemize}

%(Robin Murphy paper?) 

\textbf{Normal/Off-Normal Situation:} The normal/off-normal category describes whether the operating conditions for the mission are normal or off-normal(cite Robin Murphy paper here). 

\textbf{Mobility constraints:} The mobility constraints category describes whether team members' mobility will be restricted due to environmental features. For example, if terrain is highly variable, it might be more difficult for a team member to navigate than if it is uniform and smooth. We define three categories for mobility constraints:
\begin{itemize}
    \item None
    \item Partial
    \item Full
\end{itemize}

\textbf{Perception constraints:} Similar to the mobility constraints category, the perception constraints category describes whether team members' perception capabilities will be restricted due to environmental factors. One example of restricted perception would be limited use of cameras in foggy weather. As with mobility constraints, we define three categories for perception constraints:
\begin{itemize}
    \item None
    \item Partial
    \item Full
\end{itemize}

\textbf{Communication constraints:} The communication constraints, like mobility and perception constraints, determines whether communication will be restricted due to environmental factors. One example of this might be restricted communications in an environment with limited bandwidth. Like with mobility and perception constraints, we define three categories for communication constraints:
\begin{itemize}
    \item None
    \item Partial
    \item Full
\end{itemize}

\subsection{Team Factors - Context}
Mental models: I think we said this would be captured with a-priori SA?

Capabilities? We still need buckets for these things.

%\subsection{Task Factors - Dynamics}

\subsection{Effects}
\textbf{Load:}\\
\textbf{Cognitive load:} Given the contextual factors and chosen dynamics, each team member will have given cognitive requirements for performing the tasks that they are assigned. For autonomous agents, these are computational requirements, and for humans, these are simply cognitive processing requirements. Performance decrements may occur if a teammate is cognitively overloaded and experiences perceptual narrowing (cite). However, according to the Yerkes-Dodson law (cite here), performance can also suffer if a person is under-loaded cognitively and experiences attention lapses. Therefore, we define three categories for cognitive load:
\begin{itemize}
    \item Under-loaded
    \item Appropriately loaded
    \item Overloaded
\end{itemize}

\textbf{Physical load:} Given the contextual factors and chosen dynamics, each team member will have given physical requirements for performing their tasks in addition to the cognitive requirements. Team members need to be physically capable enough to perform the tasks they are assigned, or else they will need to ask for and wait on help from other team members. For example, if a robot can carry a 10 kg payload, if it is assigned a task that involves carrying something that weighs 20 kg, it will depend on other teammates to perform that task and will need to ask for help. We therefore define two categories to describe physical load:
\begin{itemize}
    \item Appropriate loading
    \item Overloading
\end{itemize}



\printbibliography


\end{document}
