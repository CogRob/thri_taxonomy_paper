\documentclass[letterpaper, 10 pt, conference]{ieeeconf}  % Comment this line out
                                                          % if you need a4paper
%\documentclass[a4paper, 10pt, conference]{ieeeconf}      % Use this line for a4
                                                          % paper

\IEEEoverridecommandlockouts                              % This command is only
                                                          % needed if you want to
                                                          % use the \thanks command
\overrideIEEEmargins
\let\labelindent\relax

% See the \addtolength command later in the file to balance the column lengths
% on the last page of the document



% The following packages can be found on http:\\www.ctan.org
\usepackage{graphics} % for pdf, bitmapped graphics files
%\usepackage{epsfig} % for postscript graphics files
%\usepackage{mathptmx} % assumes new font selection scheme installed
%\usepackage{times} % assumes new font selection scheme installed
%\usepackage{amsmath} % assumes amsmath package installed
%\usepackage{amssymb}  % assumes amsmath package installed
\let\proof\relax
\let\endproof\relax
\usepackage{amsthm}
\theoremstyle{definition}
\newtheorem{definition}{Definition}
\graphicspath{{images/}}
\usepackage{xcolor,soul}
\usepackage[inline, shortlabels]{enumitem}
\usepackage{biblatex}
\renewcommand*{\bibfont}{\small}
% personal productivity
\usepackage[colorinlistoftodos]{todonotes}
\presetkeys{todonotes}{inline}{}
\usepackage{hyperref}
\newcommand{\toread}[1]{\todo[color=red!40!white]{#1}}
\newcommand{\towrite}[1]{\todo[color=blue!40!white]{#1}}
\renewcommand{\hl}[1]{{\color{red}#1}}
\AtEveryBibitem{%
\ifentrytype{book}{
    \clearfield{url}%
    \clearfield{doi}%
    \clearfield{issn}%
    \clearfield{urldate}%
    \clearfield{review}%
    \clearfield{series}%%
}{}
\ifentrytype{collection}{
    \clearfield{url}%
    \clearfield{doi}%
    \clearfield{issn}%
    \clearfield{urldate}%
    \clearfield{review}%
    \clearfield{series}%%
}{}
\ifentrytype{incollection}{
    \clearfield{url}%
    \clearfield{doi}%
    \clearfield{issn}%
    \clearfield{urldate}%
    \clearfield{review}%
    \clearfield{series}%%
}{}
\ifentrytype{article}{
    \clearfield{url}%
    \clearfield{doi}%
    \clearfield{issn}%
    \clearfield{urldate}%
    \clearfield{review}%
    \clearfield{series}%%
}{}
\ifentrytype{inproceedings}{
    \clearfield{url}%
    \clearfield{doi}%
    \clearfield{issn}%
    \clearfield{urldate}%
    \clearfield{review}%
    \clearfield{series}%%
}{}
\ifentrytype{techreport}{
    \clearfield{url}%
    \clearfield{doi}%
    \clearfield{issn}%
    \clearfield{urldate}%
    \clearfield{review}%
    \clearfield{series}%%
}{}
\ifentrytype{misc}{
    \clearfield{url}%
    \clearfield{doi}%
    \clearfield{issn}%
    \clearfield{urldate}%
    \clearfield{review}%
    \clearfield{series}%%
}{}
}

\bibliography{rexam.bib}

\title{\LARGE \bf
A Taxonomy for Characterizing Modes of Interactions in Goal-driven, Human-robot Teams
}
\newcommand{\citet}[1]{\citeauthor{#1}~\cite{#1}}
%\author{ \parbox{3 in}{\centering Huibert Kwakernaak*
%         \thanks{*Use the $\backslash$thanks command to put information here}\\
%         Faculty of Electrical Engineering, Mathematics and Computer Science\\
%         University of Twente\\
%         7500 AE Enschede, The Netherlands\\
%         {\tt\small h.kwakernaak@autsubmit.com}}
%         \hspace*{ 0.5 in}
%         \parbox{3 in}{ \centering Pradeep Misra**
%         \thanks{**The footnote marks may be inserted manually}\\
%        Department of Electrical Engineering \\
%         Wright State University\\
%         Dayton, OH 45435, USA\\
%         {\tt\small pmisra@cs.wright.edu}}
%}

\author{
Priyam Parashar, Lindsay M. Sanneman, Julie A. Shah, Henrik I. Christensen
}


\begin{document}



\maketitle
\thispagestyle{empty}
\pagestyle{empty}


%%%%%%%%%%%%%%%%%%%%%%%%%%%%%%%%%%%%%%%%%%%%%%%%%%%%%%%%%%%%%%%%%%%%%%%%%%%%%%%%
\begin{abstract}
This document contains the edits to the taxonomy as well as the changes to a
priori sa framework. Additionally, the a priori sa is then further intertwined
into the dynamic factors of the taxonomy to create a more grounded framework.
\end{abstract}


%%%%%%%%%%%%%%%%%%%%%%%%%%%%%%%%%%%%%%%%%%%%%%%%%%%%%%%%%%%%%%%%%%%%%%%%%%%%%%%%

\section{Edits to Taxonomy}

\subsection{Additions}

The main article that I found and had material contributions to the taxonomy is
the one by \citet{moya2007towards}. The taxonomy takes a holistic view of a
multi-system environment and considers the following three components as overall
description of the system:
\begin{itemize}{}
  \item Situated environment
    \begin{itemize}{}
        \item Closure - If agents defined outside the system can still affect
        the environment
        \item Dynamism - Whether the environment is affected by just the system
        (static) or because of randomness/open environment (dynamic)
        \item Determinism - If actions always have a deterministic outcome over
        time, sort of like environment stationarity
        \item Cardinality - Size of environment, affected agents and objects
        (finite, finite uncountable, infinite, etc.)
    \end{itemize}
    \item Population
    \begin{itemize}{}
      \item Size
      \item Diversity
      \item Homogeneity
      \item Goal structure
      \item Cooperability
    \end{itemize}
  \item Agent characteristics
    \begin{itemize}{}
      \item Reasoning
      \item Perception
      \item Action
    \end{itemize}
\end{itemize}

I found multiple evidence in literature that capabilities are
always abstracted one way or the other. We can leverage this trend to propose
mobile, perception and reasoning bins which if present/absent represent an
agent's capabilities. TK: references of papers I found this in. The paper
mentioned in the previous paragraph also has a taxonomy spelling out bins for
perception and communication categories.

To bring back the papers we have already included, I thought the \texttt{phys_proximity}
categories in \citet{yanco2004updated} made a lot of sense for associating the
environment or workspace configurations to the physical nature of tasks.

\subsection{Changes}
\begin{itemize}{}
  \item ``Criticality'' should be further broken down into risk versus mission
    criticality. Mission criticality can be grounded into Lindsay's co-design
    method while the risk scale can be better informed by robot risk scales.
\end{itemize}

\section{A Priori Situational Awareness Framework for Stereotyping Human Roles}

\end{document}
