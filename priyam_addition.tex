\documentclass[letterpaper, 10 pt, conference]{ieeeconf}  % Comment this line out
                                                          % if you need a4paper
%\documentclass[a4paper, 10pt, conference]{ieeeconf}      % Use this line for a4
                                                          % paper

\IEEEoverridecommandlockouts                              % This command is only
                                                          % needed if you want to
                                                          % use the \thanks command
\overrideIEEEmargins
\let\labelindent\relax

% See the \addtolength command later in the file to balance the column lengths
% on the last page of the document



% The following packages can be found on http:\\www.ctan.org
\usepackage{graphics} % for pdf, bitmapped graphics files
%\usepackage{epsfig} % for postscript graphics files
%\usepackage{mathptmx} % assumes new font selection scheme installed
%\usepackage{times} % assumes new font selection scheme installed
%\usepackage{amsmath} % assumes amsmath package installed
%\usepackage{amssymb}  % assumes amsmath package installed
\let\proof\relax
\let\endproof\relax
\usepackage{amsthm}
\theoremstyle{definition}
\newtheorem{definition}{Definition}
\graphicspath{{images/}}
\usepackage{xcolor,soul}
\usepackage[inline, shortlabels]{enumitem}
\usepackage{biblatex}
\renewcommand*{\bibfont}{\small}
% personal productivity
\usepackage[colorinlistoftodos]{todonotes}
\presetkeys{todonotes}{inline}{}
\usepackage{hyperref}
\newcommand{\toread}[1]{\todo[color=red!40!white]{#1}}
\newcommand{\towrite}[1]{\todo[color=blue!40!white]{#1}}
\renewcommand{\hl}[1]{{\color{red}#1}}
\AtEveryBibitem{%
\ifentrytype{book}{
    \clearfield{url}%
    \clearfield{doi}%
    \clearfield{issn}%
    \clearfield{urldate}%
    \clearfield{review}%
    \clearfield{series}%%
}{}
\ifentrytype{collection}{
    \clearfield{url}%
    \clearfield{doi}%
    \clearfield{issn}%
    \clearfield{urldate}%
    \clearfield{review}%
    \clearfield{series}%%
}{}
\ifentrytype{incollection}{
    \clearfield{url}%
    \clearfield{doi}%
    \clearfield{issn}%
    \clearfield{urldate}%
    \clearfield{review}%
    \clearfield{series}%%
}{}
\ifentrytype{article}{
    \clearfield{url}%
    \clearfield{doi}%
    \clearfield{issn}%
    \clearfield{urldate}%
    \clearfield{review}%
    \clearfield{series}%%
}{}
\ifentrytype{inproceedings}{
    \clearfield{url}%
    \clearfield{doi}%
    \clearfield{issn}%
    \clearfield{urldate}%
    \clearfield{review}%
    \clearfield{series}%%
}{}
\ifentrytype{techreport}{
    \clearfield{url}%
    \clearfield{doi}%
    \clearfield{issn}%
    \clearfield{urldate}%
    \clearfield{review}%
    \clearfield{series}%%
}{}
\ifentrytype{misc}{
    \clearfield{url}%
    \clearfield{doi}%
    \clearfield{issn}%
    \clearfield{urldate}%
    \clearfield{review}%
    \clearfield{series}%%
}{}
}

\bibliography{rexam.bib}

\title{\LARGE \bf
A Taxonomy for Characterizing Modes of Interactions in Goal-driven, Human-robot Teams
}
\newcommand{\citet}[1]{\citeauthor{#1}~\cite{#1}}
%\author{ \parbox{3 in}{\centering Huibert Kwakernaak*
%         \thanks{*Use the $\backslash$thanks command to put information here}\\
%         Faculty of Electrical Engineering, Mathematics and Computer Science\\
%         University of Twente\\
%         7500 AE Enschede, The Netherlands\\
%         {\tt\small h.kwakernaak@autsubmit.com}}
%         \hspace*{ 0.5 in}
%         \parbox{3 in}{ \centering Pradeep Misra**
%         \thanks{**The footnote marks may be inserted manually}\\
%        Department of Electrical Engineering \\
%         Wright State University\\
%         Dayton, OH 45435, USA\\
%         {\tt\small pmisra@cs.wright.edu}}
%}

\author{
Priyam Parashar, Lindsay M. Sanneman, Julie A. Shah, Henrik I. Christensen
}


\begin{document}



\maketitle
\thispagestyle{empty}
\pagestyle{empty}


%%%%%%%%%%%%%%%%%%%%%%%%%%%%%%%%%%%%%%%%%%%%%%%%%%%%%%%%%%%%%%%%%%%%%%%%%%%%%%%%
\begin{abstract}
This document contains the edits to the taxonomy as well as the changes to a priori sa framework.
Additionally, the a priori sa is then further intertwined into the dynamic factors of the
taxonomy to create a more grounded framework.
\end{abstract}


%%%%%%%%%%%%%%%%%%%%%%%%%%%%%%%%%%%%%%%%%%%%%%%%%%%%%%%%%%%%%%%%%%%%%%%%%%%%%%%%

\section{Edits to Taxonomy}

\subsection{Additions}

The main article that I found and had material contributions to the taxonomy is the one by
\citet{moya2007towards}. The taxonomy takes a holistic view of a multi-system environment and
considers the following three components as overall description of the system:
\begin{itemize}{}
    \item Situated environment
    \begin{itemize}{}
        \item Closure - If agents defined outside the system can still affect the environment
        \item Dynamism - Whether the environment is affected by just the system (static) or because
        of randomness/open environment (dynamic)
        \item Determinism - If actions always have a deterministic outcome over time, sort of like
        environment stationarity
        \item Cardinality - Size of environment, affected agents and objects (finite, finite
        uncountable, infinite, etc.)
    \end{itemize}
    \item Population
    \begin{itemize}{}
        \item Size
        \item Diversity
        \item Homogeneity
        \item Goal structure 
        \item Cooperability 
    \end{itemize} 
    \item Agent characteristics
    \begin{itemize}{} 
    \item Reasoning 
    \item Perception 
    \item Action 
    \end{itemize} 
\end{itemize}

I found multiple evidence in the literature that capabilities are always abstracted one way or
the other. We can leverage this trend to propose mobile, perception and reasoning bins which if
present/absent represent an agent's capabilities. TK: references of papers I found this in. The
paper mentioned in the previous paragraph also has a taxonomy spelling out bins for perception
and communication categories.

To bring back the papers we have already included, I thought the \texttt{phys_proximity}
categories in \citet{yanco2004updated} made a lot of sense for associating the environment or
workspace configurations to the physical nature of tasks.

\subsection{Changes} \begin{itemize}{} \item ``Criticality'' should be further broken down into
risk versus mission criticality. Mission criticality can be grounded into Lindsay's co-design
method while the risk scale can be better informed by robot risk scales. \item Another thing I
realized was that environment is too broad a category, maybe that is why we never saw much
specification for it. Might we change ours to ``situated workspace''. I think this terminology
has better relation to what we actually define in our taxonomy. \item Given that levels of
autonomy divides levels between sensing, acting and planning, we should maybe expand the
Cognitive Task Nature category further into perception and reasoning. \item We should also use a
running example (maybe one of the two from the case-study section) to ground our abstract
statements into something intuitive. Maybe we can use the complex capture-the-flag example as the
main running example while the rest is just added to provide different viewpoints from different
domain perspectives. \end{itemize}

\section{A Priori Situational Awareness Framework for Stereotyping Human Roles}

This section introduces a bridging framework, grounded within the taxonomic factors defined in
the previous section, which helps in assessing if the system design fulfills the mode of
interaction that is established between the human and the robot. We do this by stereotyping the
expected knowledge levels of the human collaborator, stereotyped in a certain role within the
boundaries of the specified system, and compare this with the required knowledge associated with
the successful execution of the interaction relationship. We categorize these knowledge levels
along the same main axes which we used to establish our taxonomy, i.e., task-work knowledge,
team-work knowledge, and environment/situated-workspace knowledge. As we mentioned earlier, the
axes defined are not entirely independent of each other, there is some effect either of scope or
of correlation which exists between the task being done and the workspace or the team choices for
it, or vice versa. This section helps in understanding these relationships and organizes them in
a tier-based framework of knowledge levels.

Depending upon the `focus` of the task and whether the task is operating in normal or subnominal
state, one can establish the `nature` of interaction that will be required to complete the
task-work. Let's consider the capture-the-flag example the area coverage sub-task is one which
requires mainly physical cooperation within the team. On the other hand, a defend-the-flag
sub-task, which is of physical nature when a strategy is already agreed upon, can switch to a
sub-nominal state when the defense perimeter is breached and is elevated to a cognitive task
where a new strategy needs to be decided on before being physically actuated by the players. A
change in the nature of team-work inevitably entails a subsequent change in the nature of
communication between the players, which means a different set of sensing and reasoning capabilities
are required for the changed sub-task. The design of an HRI system should be able to satisfy the
largest such set of mobile, perception, and communication capabilities as well as the infrastructure
to establish the required information exchange protocols. In a very similar way, depending upon the
team-work between a human and a robot depending upon if the interaction is established with a peer
or a supervisor and whether it is in nominal or sub-nominal status, a similar set of rules dictate a
necessitated change in the quality of information being exchanged between the two interactants. This
section stereotypes the roles defined in \citet{Goodrich2007,Scholtz2003} by grounding their
expected knowledge levels for interactions to hold up in nominal conditions. This also creates a
hard boundary of what can be considered a sub-nominal state of interaction and might require a
change in the edges of the interaction graph where interactants are nodes and an edge exists
between two nodes if they are interacting.


\subsection{Julie's Comments on A Priori SA Section}

Just to keep track

\begin{center} \begin{tabular}{c | c} \hline Does this task axis related to "Knowledge" or he
task? Do you mean tactical knowledge on how to perform a task? When I saw the category task I
thought you would be defining characteristics of the task, rather than knowledge for different
ways to do it & This makes me think we should have a better introduction to what does the A
priori SA mean and does in the current scope \\ Multiple comments on clarity of role definition &
This draft will focus on this \\ Why is a certain knowledge level lowest or highest? & There
should be either a diagram or some reference to the taxonomy to make this more intuitive \\
\end{tabular} \end{center}

\end{document}
